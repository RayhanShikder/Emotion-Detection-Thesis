
\begin{thebibliography}{9}



\bibitem{ioannou} 
S. Ioannou, A. Raouzaiou, V. Tzouvaras, T. Mailis, K. Karpouzis and S. Kollias: "Emotion Recognition through Facial Expression Analysis Based on a Neurofuzzy Method",
\textit{ Neural Networks, }
vol. 18, pp. 423-435, 2005.



\bibitem{kong} 
 S.G. Kong, J. Heo, B. Abidi, J. Paik and M. Abidi: "Recent advances in visual and infrared face recognition—a review"
 \textit{ Computer Vision and Image Understanding,}
 vol. 97, no. 1, pp. 103-135, Jan. 2005.
 
 

\bibitem{gunes} 
H. Gunes and M. Pantic, \& ldquo, “Dimensional Emotion Prediction from Spontaneous Head Gestures for Interaction with Sensitive Artificial Listeners”,
\textit{ Proc. Int',l Conf. Intelligent  
     Virtual Agents,}
pp. 371-377, 2010. 

\bibitem{kwon} 
O. W. Kwon, K. Chan, J. Hao, and T. W. Lee: “Emotion recognition by speech signals,” In Proc. EUROSPEECH, 2003.


\bibitem{picard} 
Picard, R.W. 
\textit{Affective Computing.}
 MIT Press, Cambridge, 2007.
 
\bibitem{ekman} 
Ekman, P. 
\textit{Basic Emotions.}
John Wiley and Sons, Ltd, 2005.

\bibitem{russell} 
Russell, J. Core affect and the psychological
construction of emotion. 
\textit{ Psychological Review 110,}
 1
(2003), 145-172.

\bibitem{silva} 
De Silva, L. and Suen Chun, H. Real-time facial
feature extraction and emotion recognition.
\textit{ Infor.,
Comm., and Sig. Proc. 2003 and the 4th Pac. Rim
Conf. on Multimedia,}
(2003).

\bibitem{khan} 
Khan, M.M., Ingleby, M., and Ward, R.D. Automated
facial expression classification and affect interpretation
using infrared measurement of facial skin temperature
variations.
\textit{ ACM Trans. Auton. Adapt. Syst. 1,}1 (2006),
91-113.


\bibitem{partala} 
Partala, T., Surakka, V., and Vanhala, T. Real-time
estimation of emotional experiences from facial
expressions.
\textit{ Interact. Comput. 18,}2 (2006), 208-226.


\bibitem{fairclough} 
Fairclough, S. Fundamentals of physiological
computing.
\textit{ Inter. with Comp. 21,}1-2 (2009), 133-145.


\bibitem{mandryk} 
Mandryk, R.L. and Atkins, M.S. A fuzzy physiological
approach for continuously modeling emotion during
interaction with play technologies.
\textit{ Int. J. Hum.-
Comput. Stud. 65,} 4 (2007), 329-347.


\bibitem{ward} 
Ward, R.D. and Marsden, P.H. Physiological responses
to different web page designs.
\textit{ Int. J. Hum.-Comput.
Stud. 59, } 1-2 (2003), 199-212.


\bibitem{chen} 
Chen, D. and Vertegaal, R. Using mental load for
managing interruptions in physiologically attentive
user interfaces.
\textit{ CHI '04 ext. abst. on Human fac. in
comp. systems, } ACM (2004), 1513-1516.


\bibitem{stern} 
Stern, R.M., Ray, W.J., and Quigley, K.S.
\textit{ Psychophysiological recording. } Oxford University
Press, New York, 2001.


\bibitem{joyce} 
Joyce, R. and Gupta, G. Identity authentication based
on keystroke latencies.
\textit{ Commun. ACM 33, } 2 (1990),
168-176.

\bibitem{dowland} 
Dowland, P. and Furnell, S. A Long-term trial of
keystroke profiling using digraph, trigraph, and
keyword latencies. In
\textit{ IFIP Intern. Fed. for Infor.
Processing. } Springer Boston, 2004, 275-289.

\bibitem{monrose} 
Monrose, F. and Rubin, A.D. Keystroke dynamics as a biometric for authentication.
\textit{ Future Gener. Comput.
Syst. 16, } 4 (2000), 351-359.


\bibitem{bender} 
Bender, S. and Postley, H. Key sequence rhythm
recognition system and method. .

\bibitem{admit} 
Admit One Security.
\textit{ AdmitOneSecurity. } http://www.admitonesecurity.com.


\bibitem{epp} 
Epp, C. Identifying emotional states through keystroke
dynamics. 2010.http://library2.usask.ca/theses/available/etd-08312010-
131027/.


\bibitem{gunetti} 
Gunetti, D. and Picardi, C. Keystroke analysis of free
text.
\textit{ ACM Trans. Inf. Syst. Secur. 8, } 3 (2005), 312-347.


\bibitem{bergadano} 
Bergadano, F., Gunetti, D., and Picardi, C. User
authentication through keystroke dynamics.
\textit{ ACM
Trans. Inf. Syst. Secur. 5, } 4 (2002), 367-397.


\bibitem{gaines} 
Gaines, R., Lisowski, W., Press, S., and Shapiro, N.
Authentication by keystroke timing: some preliminary
results. 1980.


\bibitem{gunetti2} 
Gunetti, D., Picardi, C., and Ruffo, G. Keystroke
analysis of different languages: a case study. In 
\textit{ Lecture Notes in Computer Science. } Springer Berline /
Heidelberg, 2005, 133-144.


\bibitem{brown} 
Brown, M. and Rogers, S.J. User identification via
keystroke characteristics of typed names using neural networks.
\textit{ Int. J. Man-Mach. Stud. 39, } 6 (1993), 999-1014.

\bibitem{sheng} 
Sheng, Y., Phoha, V., and Rovnyak, S. A parallel
decision tree-based method for user authentication
based on keystroke patterns.
\textit{ IEEE Transactions on
Systems, Man, and Cybernetics 35, } 4 (2005), 826-833.


\bibitem{lang} 
Lang, P. Behavioral treatment and bio-behavioral
assessment: computer applications.
\textit{ Technology in
mental health care delivery systems, } (1980), 119-137.



\bibitem{zimmermann} 
Zimmermann, P., Guttormsen, S., Danuser, B., and
Gomez, P. Affective computing - a rationale for
measuring mood with mouse and keyboard.
\textit{ Inter. J. of
Occ. Saf. and Ergo. 9, } 4 (2003), 539-551.



\bibitem{vizer} 
Vizer, L.M., Zhou, L., and Sears, A. Automated stress
detection using keystroke and linguistic features: An
exploratory study.
\textit{ Int. J. Hum.-Comput. Stud. 67, } 10
(2009), 870-886.



\end{thebibliography}